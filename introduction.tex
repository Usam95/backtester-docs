\chapter{Introduction}

\section{Motivation}
Trading involves buying and selling assets like stocks and cryptocurrencies to make a profit. With technology's evolution, trading has also advanced, using computers to make trades based on algorithms that analyze market data. These algorithms identify optimal buy or sell moments, benefiting from market trends or inefficiencies. A major advantage of this method is its precision, as computers can process vast data quickly and make consistent, emotion-free decisions. Integrating this with cloud technology enables a 24/7 trading system that can autonomously execute trades in various markets without human intervention.

Backtesting is the process of testing a trading strategy using historical data, allowing traders to evaluate and refine their strategies before they use them in real trading. Driven by my hypothesis, questioning if algo trading strategies can indeed be beneficial, I found myself investing a significant amount of my personal time diving into the topic of algorithmic trading. Armed with the knowledge I gained and leveraging my data science and software engineering expertise, I started on building this backtesting framework. Throughout this journey, I wrote more than 7,000 lines of code to create a modular and customizable framework designed for evaluating various trading strategies, covering both technical indicators and machine learning methods.

This framework leverages the vectorized approach of Python, facilitating compact code, faster execution compared to conventional Python loops, and efficient handling of time series data. Such an approach is particularly advantageous in financial algorithm implementations, especially when it comes to vectorized backtesting.

The next section provides a brief overview of the main features of the framework. Detailed explanations of these features are provided in the following chapters.


\section{Key Highlights}

\begin{itemize}
    \item \textbf{Efficient Backtesting}: The project utilizes vectorized backtesting algorithms for testing and optimization of both classical technical indicator-based and machine learning strategies. It incorporates pre-existing technical indicators such as MACD, EMA, RSI, BB, SO, and RS, along with a variety of machine learning classification algorithms like logistic regression, SVM, and tree-based algorithms (e.g., Random Forest, AdaBoost, Gradient Boosting Classifier). Moreover, the system is architected for easy extensibility, allowing the integration of new algorithms without altering previously implemented and validated code. \newpage

    \item \textbf{Data Source}: The data for this project is directly sourced from Binance using the \textit{data\_retriever} module, primarily in 1-minute intervals. This data can be downsampled to longer intervals, allowing backtests over extended periods when necessary.

    \item \textbf{Dynamic Optimization}: Both the implemented technical indicators and machine learning models can be tested and optimized based on various performance metrics. Parameter spaces for optimization can be easily configured for each strategy in the dedicated JSON configuration file. Currently, the optimization supports GridSearch for both technical indicators and machine learning, as well as Bayesian Optimization for technical indicators. Support for additional optimization techniques will be added in the future.

    \item \textbf{User-centric Configurations}: As previously mentioned, the optimization of technical indicators and machine learning models is supported through configuration files. User-friendly configurations can also be set up for deployment, where the algorithms to be used, the symbols to be traded, and various other parameters can be specified.

    \item \textbf{Deployment Simplified}: Post-backtesting, the fine-tuned strategies are primed for real trading sessions. The trading infrastructure is also designed to be adaptable and can interface with any broker that provides a trading API. At present, a Binance client is incorporated into the system. To accommodate new brokers, a distinct module would need to be developed, leveraging the interfaces for strategies and the data manager that supplies the trading signals. The system is engineered to support multiple strategies and symbols concurrently.

    \item \textbf{Real-time Notifications}: Thanks to integrated logging mechanisms and email notifications, users remain consistently informed regarding backtesting outcomes and live trading activities.

\end{itemize}


%As shown in Figure~\ref{fig:projectStructure}, the project is organized into several directories...


\chapter{Introduction}

\section{Motivation}
Trading, in its simplest form, is the act of buying and selling assets, like stocks or commodities, with the aim of making a profit. As technology has advanced, so too has the approach to trading. This has given rise to what's known as algorithmic trading, where computers are programmed to make trade decisions based on a set of predefined criteria or algorithms. These algorithms analyze market data to determine the ideal buying or selling points, aiming to capitalize on market inefficiencies or trends.

One of the standout benefits of algorithmic trading is precision. Computers can analyze vast amounts of data at unparalleled speeds and execute trades in fractions of a second. This speed and efficiency often lead to more profitable trading opportunities. Additionally, by removing human emotions from the trading process, algorithmic trading often results in more rational, consistent decision-making.

When you couple algorithmic trading with cloud technology, you gain the advantage of a trading system that operates 24/7, without the constraints of a physical location. An automated trading bot running in the cloud can monitor multiple markets, execute trades in real-time, and even adapt to changing market conditions, all while requiring minimal human intervention. Such a bot can send you notifications about its activities, ensuring you're always in the loop without being attached to a trading desk.


\textbf{Inspired by the transformative potential of algorithmic trading and recognizing its numerous advantages, I have created the "Algo Trading with Backtesting" project}. This project is a testament to that belief, combining the strengths of algorithmic trading, machine learning, and data science into a unified system. It's more than just code—it encapsulates countless hours of diverse research
, trial, and refinement of various trading strategies. An integral part of this endeavor wasn't merely crafting these strategies but fine-tuning them for optimum performance, ensuring compatibility with both technical indicators and sophisticated machine learning models. To make things accessible and user-friendly, the configuration of various backtesting aspects have been employed. This allows for the easy setting of parameters, defining of parameter spaces, and other crucial configurations.
For those with an appetite for deeper exploration, the full source code and repository await on GitHub, aptly named "Algo Trading Backtester."

\section{Key Highlights}

\begin{itemize}
    \item \textbf{Efficient Backtesting}: The project utilizes vectorized backtesting algorithms for testing and optimization of both classical technical indicator-based and machine learning strategies. It incorporates pre-existing technical indicators such as MACD, EMA, RSI, BB, SO, and RS, along with a variety of machine learning classification algorithms like logistic regression, SVM, and tree-based algorithms (e.g., Random Forest, AdaBoost, Gradient Boosting Classifier). Moreover, the system is architected for easy extensibility, allowing the integration of new algorithms without altering previously implemented and validated code. \newpage

    \item \textbf{Data Source}: The data for this project is directly sourced from Binance using the \textit{data\_retriever} module, primarily in 1-minute intervals. This data can be downsampled to longer intervals, allowing backtests over extended periods when necessary.

    \item \textbf{Dynamic Optimization}: Both the implemented technical indicators and machine learning models can be tested and optimized based on various performance metrics. Parameter spaces for optimization can be easily configured for each strategy in the dedicated JSON configuration file. Currently, the optimization supports GridSearch for both technical indicators and machine learning, as well as Bayesian Optimization for technical indicators. Support for additional optimization techniques will be added in the future.

    \item \textbf{User-centric Configurations}: As previously mentioned, the optimization of technical indicators and machine learning models is supported through configuration files. User-friendly configurations can also be set up for deployment, where the algorithms to be used, the symbols to be traded, and various other parameters can be specified.

    \item \textbf{Deployment Simplified}: Post-backtesting, the fine-tuned strategies are primed for real trading sessions. The trading infrastructure is also designed to be adaptable and can interface with any broker that provides a trading API. At present, a Binance client is incorporated into the system. To accommodate new brokers, a distinct module would need to be developed, leveraging the interfaces for strategies and the data manager that supplies the trading signals. The system is engineered to support multiple strategies and symbols concurrently.

    \item \textbf{Real-time Notifications}: Thanks to integrated logging mechanisms and email notifications, users remain consistently informed regarding backtesting outcomes and live trading activities.

\end{itemize}

\begin{figure}[h]
\dirtree{%
.1 backtester.
.2 historical\_data.
.3 BTCUSDT
.3 ETHUSDT
.3 \ldots.
.2 \ldots.
.2 utilities.
.3 credentials.py.
.3 data\_plot\_ml.py.
.3 logger.py.
.3 performance.py.
.3 report\_email.py.
.3 data\_utils.
.4 data\_loader.py.
.4 data\_manager.py.
.4 data\_retriever.py.
.4 ml\_data\_manager.py.
.4 ml\_feature\_engineer.py.
.3 {\textbf{plot\_utils}}.
.4 {\textbf{backtesting\_plotter.py}}.
.4 {\textbf{ml\_model\_evaluator.py}}.
}

\caption{Directory structure of the project.}\label{fig:projectStructure}
\end{figure}

%As shown in Figure~\ref{fig:projectStructure}, the project is organized into several directories...

